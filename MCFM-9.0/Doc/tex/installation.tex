\section{Installation}
\label{sec:Installation}

The \MCFM{} package may be downloaded from the \MCFM{} homepage at \url{https://mcfm.fnal.gov}.
After extracting, the source can be compiled by first running the
{\tt ./Install} command, which compiles some tensor reduction components
as well as the included libraries QCDLoop and QD. Then the main part can be compiled with {\tt make} (we require GNU 
make).
By default, the Fortran and C compilers from GCC are used, but we also allow for the
Intel Fortran and C compilers. These can be enabled by setting the flag
{\tt USEINTEL} in the Install and makefile files. For the GNU compiler
collection at least version 7 is required.

If your GNU Fortran and GNU C++ compiler commands are different than gfortran
and g++ you should export the environment variables \texttt{FC} and \texttt{CXX} to match the
names of the gfortran and g++ compiler commands before running the Install
script or the makefile. These variables can also be adjusted in the Install and makefile files.

Additional complications may arise on OS X systems, where by default gcc/g++
is linked to the clang compiler. Please make sure that the g++ command used
in the files Install and makefile correspond to the actual GNU g++ compiler command,
or export the \texttt{FC} and \texttt{CXX} environment variables.

To enable the use of \LHAPDF{}, the variable \texttt{PDFROUTINES} in the makefile should be
set to \texttt{LHAPDF}, and the path \texttt{LHAPDFLIB} can be adjusted to the path containing
\texttt{libLHAPDF.so}. This version of \MCFM{} has been 
explicitly tested against {\tt LHAPDF-6.2.1}. \texttt{LHAPDF} version 5 is not supported.

If at the first execution of \MCFM{} the library libLHAPDF cannot be found, you can add your \LHAPDF{} library 
path to the environment variable {\tt LD\_LIBRARY\_PATH} as follows:
\begin{verbatim}
export LD_LIBRARY_PATH=${LD_LIBRARY_PATH}:/home/user/newpath/lib
\end{verbatim}

Please ensure that your compiler is working and can produce executable program
files. For example when your compiler has been installed into a non-standard
location you probably need to append the compiler library path to {\tt
LD\_LIBRARY\_PATH } ({\tt DYLD\_FALLBACK\_LIBRARY\_PATH} on OS X).  This can be
achieved, for example, as follows:
\begin{verbatim}
export LD_LIBRARY_PATH=${LD_LIBRARY_PATH}:/home/user/local/lib/gcc7
\end{verbatim}

The directory structure of \MCFM{} is as follows:
\begin{itemize}
\item {\tt Doc}. The source for this document.
\item {\tt Bin}. The directory containing the executable {\tt mcfm\_omp},
and various essential files -- notably the options file {\tt input.ini}.
\item {\tt Bin/Pdfdata}. The directory containing the internal \PDF{} data-files.
\item {\tt obj}. The object files produced by the compiler. 
\item {\tt src}. The Fortran source files in various subdirectories.
\item {\tt qcdloop-2.0.2}. The source files to the library QCDLoop~\cite{Carrazza:2016gav,Ellis:2007qk}.
\item {\tt TensorReduction} General tensor reduction code based on the work of Passarino and 
Veltman \cite{Passarino:1978jh} and Oldenborgh and Vermaseren \cite{vanOldenborgh:1989wn}.
\item {\tt qd-2.3.22}. Library to support double-double and quad-double precision data types \cite{libqd}.
\end{itemize}
 
\subsection{OpenMP and MPI}
\MCFM{} uses \OMP{} (Open Multi-Processing) to implement multi-threading and automatically adjusts to the 
number of available \CPU{} threads. The multi-threading is implemented with respect to the integration routine 
Vegas, which distributes the event evaluations over the threads and combines all events at the end of 
every iteration.

Two environment variables are useful. On some systems, depending on the \OMP{} implementation,
the program will crash when calculating some of the more complicated processes,
for example $W+2$~jet production at \NLO{}.
Then, adjusting {\tt OMP\_STACKSIZE} may be needed for the program to run correctly.
Setting thisvariable to {\tt 16000}, for instance in the Bash shell by using the
command {\tt export OMP\_STACKSIZE=16000}, has been found to be sufficient
for all processes.  The second useful variable {\tt OMP\_NUM\_THREADS}
may be used to directly control the number of threads used during
\OMP{} execution (the default is the maximum number of threads available
on the system).

It is also possible to run \MCFM{} using \MPI{} (Message Passing Interface).
To run in this mode, change the flag {\tt USEMPI} in the makefile to {\tt YES} and specify the MPI compiler 
wrappers and compilers in the makefilen or set the environment variables \texttt{FC} and \texttt{CXX}.
By default, the OpenMPI compiler wrappers mpifort and mpic++ are used,
to use gfortran and g++. When {\tt USEINTEL} is set, mpiifort and mpicc are used.
 