\section{Configuration}
\label{Input_parameters}

\subsection{Compile-time settings}
\MCFM{} allows the user to choose between a number of schemes
for defining the electroweak couplings. These choices are summarized
in Table~\ref{ewscheme}. The scheme is selected by modifying the
value of {\tt ewscheme} in {\tt src/User/mdata.f} prior to compilation, 
which also contains
the values of all input parameters (see also Table~\ref{default}).

\begin{table}
\begin{center}
	\caption{Different options for the scheme used to fix the electroweak
		parameters of the Standard Model and the corresponding default input
		values. $M_W$ and $M_Z$ are taken from ref.~\cite{Amsler:2008zzb}.}
	\label{ewscheme}
	\vspace{0.5em}
\begin{tabular}{|c|c|c|c|c|c|c|} \hline
 Parameter & Name & Input Value
 & \multicolumn{4}{c|}{Output Value determined by \tt ewscheme} \\
\cline{4-7}
& ({\tt \_inp}) & & {\tt -1} & {\tt 0} & {\tt 1} & {\tt 2} \\ \hline
$G_F$            & {\tt Gf}      & 1.16639$\times$10$^{-5}$ 
 & input & calculated & input & input \\
$\alpha(M_Z)$    & {\tt aemmz}   & 1/128.89                 
 & input & input & calculated & input \\
$\sin^2 \theta_w$& {\tt xw}      & 0.2223               
 & calculated & input & calculated & input \\
$M_W$            & {\tt wmass}   & 80.385 GeV                
 & input & calculated & input & calculated \\
$M_Z$            & {\tt zmass}   & 91.1876 GeV               
 & input & input & input & calculated \\
$m_t$            & {\tt mt}      & {\tt input.ini}                  
 & calculated & input & input & input \\
\hline
\end{tabular}
\end{center}
\end{table}

The default scheme corresponds to {\tt ewscheme=+1}. As described below, this corresponds to a scheme
in which the top quark mass is an input parameter so that it is
more suitable for many processes now included in the program.

The choice of ({\tt ewscheme=-1}) enforces the use of an effective field
theory approach, which is valid for scales below the top mass. In this
approach there are 4 independent parameters (which we choose to be
$G_F$, $\alpha(M_Z)$, $M_W$ and $M_Z$). For further details,
see Georgi~\cite{Georgi:1991ci}.

For all the other schemes ({\tt ewscheme=0,1,2}) the top mass is simply
an additional input parameter and there are 3 other independent
parameters from the remaining 5. The variable {\tt ewscheme} then performs
exactly the same role as {\tt idef} in MadEvent~\cite{Maltoni:2002qb}.
{\tt ewscheme=0} is the old MadEvent default and {\tt ewscheme=1} is the
new MadEvent default, which is also the same as that used in 
Alpgen~\cite{Alpgen} and LUSIFER~\cite{Lusifer} 
For processes in which the top quark is directly produced  it is 
preferable to use  the schemes ({\tt ewscheme=0,1,2}), since in these schemes
one can adjust the top mass to its physical value (in the input file
{\tt input.ini}).

\begin{table}
\begin{center}
	\caption{Default values for the remaining parameters in \MCFM.
		$\Gamma_W$ and $\Gamma_Z$ from ref.~\cite{Amsler:2008zzb}.}
	\label{default} 
	\vspace{0.5em}
\begin{tabular}{|c|c|c|} \hline
Parameter & Fortran name & Default value \\ 
\hline
$m_\tau$         & {\tt mtau}      & 1.777 GeV            \\
$m^2_\tau$& {\tt mtausq}  & 3.1577 GeV$^2$     \\
$\Gamma_\tau$    & {\tt tauwidth}& 2.269$\times$10$^{-12}$~GeV \\
$\Gamma_W$       & {\tt wwidth}  & 2.093 GeV               \\
$\Gamma_Z$       & {\tt zwidth}  & 2.4952 GeV               \\
$V_{ud}$         & {\tt Vud}     & 0.975                  \\
$V_{us}$         & {\tt Vus}     & 0.222             \\
$V_{ub}$         & {\tt Vub}     & 0.                     \\
$V_{cd}$         & {\tt Vcd}     & 0.222             \\
$V_{cs}$         & {\tt Vcs}     & 0.975                  \\
$V_{cb}$         & {\tt Vcb}     & 0.                     \\
\hline
\end{tabular}

\end{center}
\end{table}

% I consider this setting dangerous, where is it used?

%In the same file ({\tt mdata.f}) one can also choose the definition
%that the program uses for computing transverse quantities, namely
%transverse momentum or transverse energy. These are defined by,
%\begin{eqnarray}
%\mbox{transverse momentum:} & \sqrt{p_x^2+p_y^2} \; ,\nonumber \\
%\mbox{transverse energy:}   &
% \frac{E \sqrt{p_x^2+p_y^2}}{\sqrt{p_x^2+p_y^2+p_z^2}} \; .
%\end{eqnarray}
%The two definitions of course coincide for massless particles.
%The chosen definition is used for all cuts that are applied to the
%process and it is the one that is used in the default set of histograms.

\subsection{Parton distributions}
\label{subsec:pdfsets}
The value of $\alpha_s(M_Z)$ is not adjustable; it is hardwired with the
parton distribution. In addition, the parton distribution also specifies
the number of loops that should be used in the running of $\alpha_s$.
The default mode of operation is to choose from a
collection of modern parton distribution functions that are included with
\MCFM{}.  The distributions, together with their associated $\alpha_s(M_Z)$
values, are given in \cref{pdlabelrecent} on \cpageref{pdlabelrecent}. In addition to these
choices, a number of historical PDF sets are also available;
 for details see the manual for \MCFM{}-8.0.
Note that, 
due to the memory requirements for
using the NNPDF sets, in OpenMP operation it is usually necessary to increase
the value of the environment variable {\tt OMP\_STACKSIZE} to avoid
segmentation faults.
%
\begin{table}[h]
\begin{center}
\begin{tabular}{|c|c|c|c|}
\hline
{\tt pdlabel}  & $\alpha_s(M_Z)$ & order & reference \\
\hline
{\tt mstw8lo}  & 0.1394 & 1     & \cite{Martin:2009iq} \\
{\tt mstw8nl}  & 0.1202 & 2     & \cite{Martin:2009iq} \\
{\tt mstw8nn}  & 0.1171 & 3     & \cite{Martin:2009iq} \\
{\tt MMHT\_lo}  & 0.135  & 1     & \cite{Harland-Lang:2014zoa} \\
{\tt MMHT\_nl}  & 0.120  & 2     & \cite{Harland-Lang:2014zoa} \\
{\tt MMHT\_nn}  & 0.118  & 3     & \cite{Harland-Lang:2014zoa} \\
\hline
{\tt CT10.00}  & 0.118  & 2     & \cite{Lai:2010vv} \\
{\tt CT14.LL}  & 0.130  & 1     & \cite{Dulat:2015mca} \\
{\tt CT14.NL}  & 0.118  & 2     & \cite{Dulat:2015mca} \\
{\tt CT14.NN}  & 0.118  & 3     & \cite{Dulat:2015mca} \\
{\tt CT14qed}  & 0.118  & 2     & \cite{Schmidt:2015zda} \\
\hline
{\tt NN2.3NL}  & 0.118  & 2     & \cite{Ball:2012cx} \\
{\tt NN2.3NN}  & 0.118  & 3     & \cite{Ball:2012cx} \\
{\tt NN3.0LO}  & 0.118  & 1     & \cite{Ball:2014uwa} \\
{\tt NN3.0NL}  & 0.118  & 2     & \cite{Ball:2014uwa} \\
{\tt NN3.0NN}  & 0.118  & 3     & \cite{Ball:2014uwa} \\
\hline
\end{tabular}
\end{center}
\caption{Modern PDF sets that are available in the code,
their corresponding values of $\alpha_s(M_Z)$ and order of running,
and a reference to the paper
that describes their origin.  Further sets, of a more historical nature, are listed in \cref{olderPDFs}.
\label{pdlabelrecent}}
\end{table}

\subsection{Electroweak corrections}
\label{subsec:EW}

As of version 8.1, {\tt MCFM} allows the calculation of weak corrections to a
selection of processes: {\tt 31} (neutral-current DY),
{\tt 157} (top-pair production) and {\tt 190} (di-jet production).
This is controlled by the flag {\tt ewcorr} in the input file.  A complete description
of the calculations is provided in Ref.~\cite{Campbell:2016dks}.

By setting {\tt ewcorr} to {\tt sudakov}, the program performs a calculation of
the leading weak corrections to these processes using a Sudakov approximation that
is appropriate at high energies.   The calculation of the weak corrections using the
exact form of the one-loop amplitudes is obtained by using the flag {\tt exact}.
A comparison between the two approaches, together with discussions of the validity of
the Sudakov approximation, may be found in Ref.~\cite{Campbell:2016dks}.

For the case of top-pair and di-jet production, the weak one-loop corrections contain
infrared divergences that must be cancelled against corresponding real radiation
contributions (in much the same manner as a regular NLO QCD calculation).  For this
reason the screen output will contain two sets of iterations corresponding to the
virtual and real contributions.

For all processes, performing the calculation of weak
corrections enables a special mode of phase-space integration that is designed to
better-sample events produced at high-energies.  For this reason the VEGAS output that
appears on the screen does not correspond to a physical cross-section -- and a corresponding
warning message to this effect will be displayed.  In many cases the quantity of most interest
is the relative correction to the leading order result ($\delta_{\rm wk}$) given by,
\begin{equation}
\delta_{\rm wk} = \frac{d\sigma_{\rm wk}^{NLO} - d\sigma^{LO}}{d\sigma^{LO}} \;.
\end{equation}
It is straightforward to compute this quantity for a distribution by editing the appropriate
{\tt nplotter} routine.  This is achieved by filling a histogram with the weight corresponding
to the LO result, another with the weight for the NLO weak result and then an additional placeholder
histogram that contains the special string {\tt '+RELEW+'}.  Examples of the syntax and correct calling
sequence can be seen in the code.


\subsection{Nuclear collisions}
\label{sec:nucleus}

It is possible to specify nuclear collisions by choosing values
of {\tt ih1} and/or {\tt ih2} above {\tt 1000d0}. In that case,
the identity of the nucleus is specified by the atomic number
and mass ($Z$ and $A$ respectively) as follows:
\begin{equation}
{\tt ih} = 1000Z+A.
\end{equation}
For example, to choose an incoming lead beam one would set
{\tt ih1=+82207d0}, corresponding to $Z=82$ and $A=207$.
When running the program, the value of {\tt sqrts} should also be
changed. This must be done by hand and is not automatically taken
care of by the
program. The centre-of-mass energy is decreased by a factor of
$\sqrt{Z/A}$ for each nuclear beam. 

The nucleon \PDF{}s are calculated by applying the correction
factors of EKS98~\cite{Eskola:1998df} on top of the \PDF{} set that is selected.
This construction simply corrects each parton distribution by
a factor that depends on the value of $(x,\mu)$ in the event.
This parametrization is limited to the region $\mu < 100$~GeV and
any value above that threshold will instead default to $100$~GeV.

Note that the cross-section reported by the program at the end
of the run is given per nucleon per beam. Therefore the
appropriate factors of $A$ should be applied in order to obtain
the total cross section.


\subsection{Run-time input file configuration}

\MCFM{} execution is performed in the {\tt Bin/} directory,
with syntax:
\begin{center}
	{\tt mcfm\_omp }{\it input.ini}
\end{center}
If no command line options are given, then \MCFM{} will default
to using the file {\tt input.ini} in the current directory for
choosing options. The \texttt{input.ini} file can be in any directory and
then the first argument to \texttt{mcfm\_omp} should be the location
of the file. Furthermore, one can overwrite or append single
configuration options with additional parameters like 
\texttt{./mcfm\_omp benchmark/input.ini -general\%part=nlo -lhapdf\%dopdferrors=.true.}.
Here specifying a parameter uses a single dash, then the section name as in the configuration file, followed
by a percent sign, followed by the option name, followed by an equal sign and the actual value of the setting.

All default settings in the input file are explained below, as well as further optional parameters.
The top level setting \texttt{mcfm\_version} specifies the input file version number and it must  match the version of 
the code being used.

%\begin{table}[]
	\begin{longtable}{p{1.5cm}p{12cm}}
		\toprule
		\multicolumn{1}{c}{{\textbf{Section} \texttt{general}}} & \multicolumn{1}{c}{{\textbf{Description}}} \\ 
		\midrule
		\texttt{nproc} & 
		The process to be studied is given by
		choosing a process number, according to Table~\ref{nproctable}
		in Appendix~\ref{MCFMprocs}.
		$f(p_i)$ denotes a generic partonic jet. Processes denoted as
		``LO'' may only be calculated in the Born approximation. For photon
		processes, ``NLO+F'' signifies that the calculation may be performed
		both at NLO and also including the effects of photon fragmentation
		and experimental isolation. In contrast, ``NLO'' for a process involving
		photons means that no fragmentation contributions are included and isolation
		is performed according to the procedure of Frixione~\cite{Frixione:1998jh}.	\\
		\texttt{part} &
		This parameter has 5 possible values, described below:
		\begin{itemize}
			\item {\tt lo} (or {\tt lord}).
			The calculation is performed at leading order only.
			\item {\tt virt}.
			Virtual (loop) contributions to the next-to-leading order result are
			calculated (+counterterms to make them finite), including also the
			lowest order contribution.
			\item {\tt real}.
			In addition to the loop diagrams calculated by {\tt virt}, the full
			next-to-leading order results must include contributions from diagrams
			involving real gluon emission (-counterterms to make them finite).
			Note that only the sum of the {\tt real} and the {\tt virt} contributions
			is physical.
			\item {\tt nlo} (or {\tt tota}).
			For simplicity, the {\tt nlo} option simply runs the {\tt virt} and
			{\tt real} real pieces in series before performing a sum to obtain
			the full next-to-leading order result. For photon processes that include fragmentation,
			{\tt nlo} also includes the calculation of the fragmentation ({\tt frag})
			contributions.
			\item {\tt nlocoeff}.
			This computes only the contribution of the NLO coefficient;  it is equivalent
			to running {\tt nlo} and then subtracting the result of {\tt lo}.
			\item {\tt nlodk} (or {\tt todk}).
			Processes 114, 161, 166, 171, 176, 181, 186, 141, 146, 149, 233, 238, 501, 511 only, see 
			sections~\ref{subsec:stop} and
			\ref{subsec:wt} below.
			\item {\tt frag}.
			Processes 280, 285, 290, 295, 300-302, 305-307,  820-823 only, see sections~\ref{subsec:gamgam}, 
			\ref{subsec:wgamma} and
			\ref{subsec:zgamma} below.
			\item {\tt nnlo} (and {\tt nnlocoeff}).
			The computation of the NNLO prediction (or the NNLO coefficient in the
			expansion) is described separately below.
		\end{itemize} \\
		\texttt{runstring} &
		When \MCFM{} is run, it will write output to several files. The
		label {\tt runstring} will be included in the names of these files.
		\\
		\texttt{sqrts} & Center of mass energy in GeV. \\
		\texttt{ih1}, \texttt{ih2} &
		The identities of the incoming hadrons
		may be set with these parameters, allowing simulations for both
		$p{\bar p}$ (such as the Tevatron) and $pp$ (such as the LHC). 
		Setting {\tt ih1} equal to ${\tt +1}$ corresponds to
		a proton, whilst ${\tt -1}$ corresponds to an anti-proton.
		Values greater than {\tt 1000d0} represent a nuclear collision,
		as described in Section~\ref{sec:nucleus}. \\
		\texttt{zerowidth} &
		When set to {\tt .true.} then all vector
		bosons are produced on-shell. This is appropriate for calculations
		of {\it total} cross-sections (such as when using {\tt removebr} equal
		to {\tt .true.}, below). When interested in decay products of the
		bosons this should be set to {\tt .false.}. \\
		\texttt{removebr} &
		When set to {\tt .true.} the branching ratios are 
		removed for unstable particles such as vector bosons or top quarks. See the
		process notes in Section~\ref{sec:specific} below for further details. \\
		\texttt{ewcorr} & 
		Specifies whether or not to compute EW corrections
		for the process.  Default is {\tt none}.  May be set to {\tt exact}
		or {\tt sudakov} for processes {\tt 31} (neutral-current DY),
		{\tt 157} (top-pair production) and {\tt 190} (di-jet production).
		For more details see section~\ref{subsec:EW}.		\\
		\bottomrule
	\end{longtable}
%\end{table}

%\begin{table}[]
	\begin{longtable}{p{1.5cm}p{12cm}}
		\toprule
		\multicolumn{1}{c}{{\textbf{Section} \texttt{nnlo}}} & \multicolumn{1}{c}{{\textbf{Description}}} \\ 
		\midrule
		\texttt{taucut} & 
		Optional. This sets the value of the jettiness variable
		$\tau_\text{cut}$, as multiplied by the invariant mass of the Born system,
		that separates the resolved and unresolved regions in \NNLO{}
		calculations that use zero-jettiness. The default value results
		in total inclusive cross sections with less than $1\%$ residual cutoff effects, see \cref{sec:NNLO}. \\
		\texttt{tcutarray} &
		Optional. Array that specifies multiple taucut values that should be sampled
		on the fly in addition to the nominal taucut value. Both larger and smaller
		values than the nominal one can be specified, although uncertainties for
		smaller values will be large. We generally do not recommend smaller values
		than the nominal one chosen with \texttt{taucut}. Default values are chosen
		to be $2,4,8,20,40$ times the nominal choice of \texttt{taucut}.  \\
		\texttt{dynamictau} &
		Optional. If \texttt{.false.}, the \texttt{taucut} value specified
		is not multiplied by the invariant mass of the Born system. Default is \texttt{.true.}. \\
		\bottomrule
	\end{longtable}
%\end{table}


%\begin{table}[]
	\begin{longtable}{p{1.5cm}p{12cm}}
		\toprule
		\multicolumn{1}{c}{{\textbf{Section} \texttt{pdf}}} & \multicolumn{1}{c}{{\textbf{Description}}} \\ 
		\midrule
		\texttt{pdlabel} &
		This specifies the parton distributions used in case the code has been built with
		\texttt{PDFROUTINES = NATIVE}. The choice of parton distribution is made by
		inserting the appropriate 7-character code from the table in \cref{subsec:pdfsets}
		or in \cref{olderPDFs} for historical \PDF{} sets.
		As mentioned above, this also sets the value of $\alpha_S(M_Z)$.\\
		\bottomrule
	\end{longtable}
%\end{table}

%\begin{table}[]
	\begin{longtable}{p{1.5cm}p{12cm}}
		\toprule
		\multicolumn{1}{c}{{\textbf{Section} \texttt{lhapdf}}} & \multicolumn{1}{c}{{\textbf{Description}}} \\ 
		\midrule
		\texttt{lhapdfset} &
		Specifies the parton distributions used in case the code has been built with
		\texttt{PDFROUTINES = LHAPDF}. For a default global installation the PDFs reside
		in \texttt{/usr/share/LHAPDF/} or \texttt{/usr/local/share/LHAPDF}, and the name
		equals the set name from \url{https://lhapdf.hepforge.org/pdfsets.html}, which is
		also the directory name of the sets. Multiple PDF sets separated by a space can be specified. \\
		\texttt{lhapdfmember} & Specifies the individual members of the parton distribution sets.
		A value of zero corresponds to the central value for Hessian sets. In case multiple sets
		have been specifies above, each one needs a member number separated by space. \\
		\texttt{dopdferrors} & When this is set to \texttt{.true.} PDF uncertainties are calculated
		for every specified \PDF{} set according to the routines provided by \LHAPDF{}.
		The \texttt{lhapdfmember} numbers are ignored but must still be set for each member. \\
		\bottomrule
	\end{longtable}
%\end{table}

%\begin{table}[]
	\begin{longtable}{p{1.5cm}p{12cm}}
		\toprule
		\multicolumn{1}{c}{{\textbf{Section} \texttt{scales}}} & \multicolumn{1}{c}{{\textbf{Description}}} \\ 
		\midrule
		\texttt{renscale} &
		This parameter may be used to adjust the value
		of the {\it renormalization} scale. This is the scale
		at which $\alpha_S$ is evaluated and will typically be set to
		a mass scale appropriate to the process ($M_W$, $M_Z$, $M_t$ for
		instance). \\
		\texttt{facscale} &
		This parameter may be used to adjust the value
		of the {\it factorization} scale and will typically be set to
		a mass scale appropriate to the process ($M_W$, $M_Z$, $M_t$ for
		instance). \\
		\texttt{dynamicscale} &
		This character string is used to specify whether
		the renormalization, factorization and fragmentation scales are dynamic, i.e. recalculated
		on an event-by-event basis. If this string is set to `{\tt none}' then the scales
		are fixed for all events at the values	specified by {\tt renscale}, {\tt facscale}
		as well as \texttt{fragmentation\_scale} as defined further below.
		
		The type of dynamic scale to be used is selected by using a particular string
		for the variable {\tt dynamicscale}, as indicated in \cref{tab:dynamicscales} on \cpageref{tab:dynamicscales}.
		Not all scales are defined for each process, with program execution halted if
		an invalid selection is made in the input file.
		The selection chooses a reference scale, $\mu_0$. The actual scales used in
		the code are then,
		\begin{equation}
		\mu_{\rm ren} = {\tt scale} \times \mu_0 \;, \qquad
		\mu_{\rm fac} = {\tt facscale} \times \mu_0
		\label{eq:dynscale}
		\end{equation}
		Note that, for simplicity, the fragmentation scale (relevant only for processes
		involving photons) is set equal to the renormalization scale.
		In some cases it is possible for the dynamic scale to become very large. This can cause problems 
		with the interpolation of data tables for the PDFs and fragmentation functions. As a result if a dynamic scale 
		exceeds a maximum of $60$ TeV (PDF) or $990$ GeV (fragmentation) this value is set by default to the maximum. 	
		\\
		\texttt{doscalevar} &
		
		This additional option can be set to \texttt{.true.} to enable scale variation.
		It performs a variation of the scales used in \cref{eq:dynscale} by a factor of 
		two so that it surveys the 
		additional possibilities,
		\begin{eqnarray}
		&&
		(2\mu_{\rm ren},2\mu_{\rm fac}),
		(\mu_{\rm ren}/2,\mu_{\rm fac}/2), \nonumber \\ &&
		(2\mu_{\rm ren},\mu_{\rm fac}),
		(\mu_{\rm ren}/2,\mu_{\rm fac}),
		(\mu_{\rm ren},2\mu_{\rm fac}),
		(\mu_{\rm ren},\mu_{\rm fac}/2) \,.
		\label{eq:scalevar}
		\end{eqnarray}
		The histograms corresponding to these different choices are included in the output file, from which an
		envelope of theoretical uncertainty may be constructed by the user. \\
		\texttt{maxscalevar} &
		Number of additional scale variation points to choose, can be set to two or six. For two
		it just samples the first two variations as in eq.~\ref{eq:scalevar}. \\
		\bottomrule
	\end{longtable}
%\end{table}

\begin{table}
	\begin{center}
		\begin{longtable}{|l|l|l|}
			\hline
			{\tt dynamic scale} & $\mu_0^2$ & comments\\
			\hline 
			{\tt m(34)} & $(p_3+p_4)^2$ & \\
			{\tt m(345)} & $(p_3+p_4+p_5)^2$ & \\
			{\tt m(3456)} & $(p_3+p_4+p_5+p_6)^2$ & \\
			{\tt sqrt(M\pow 2+pt34\pow 2)} & $M^2 + (\vec{p_T}_3 + \vec{p_T}_4)^2$ & $M=$~mass of particle 3+4 \\
			{\tt sqrt(M\pow 2+pt345\pow 2)} & $M^2 + (\vec{p_T}_3 + \vec{p_T}_4 + \vec{p_T}_5)^2$ & $M=$~mass of 
			particle 3+4+5 \\
			{\tt sqrt(M\pow 2+pt5\pow 2)} & $M^2 + \vec{p_T}_5^2$ & $M=$~mass of particle 3+4 \\
			{\tt sqrt(M\pow 2+ptj1\pow 2)} & $M^2 + \vec{p_T}_{j_1}^2$ & $M=$~mass(3+4), $j_1=$ leading $p_T$ jet \\
			{\tt pt(photon)} & $\vec{p_T}_\gamma^2$ & \\
			{\tt pt(j1)} & $\vec{p_T}_{j_1}^2$ & \\
			{\tt HT} & $\sum_{i=1}^n {p_T}_i$ & $n$ particles (partons, not jets) \\
			\hline 
			\hline\end{longtable}
	\end{center}
	\caption{Choices of the input parameter {\tt dynamicscale} that result in an event-by-event
		calculation of all relevant scales using the given reference scale-squared $\mu_0^2$.
		\label{tab:dynamicscales}}
\end{table}

%\begin{table}[]
	\begin{longtable}{p{1.5cm}p{12cm}}
		\toprule
		\multicolumn{1}{c}{{\textbf{Section} \texttt{masses}}} & \multicolumn{1}{c}{{\textbf{Description}}} \\ 
		\midrule
		\texttt{hmass} & Higgs mass \\
		\texttt{mt} & Top-quark mass \\
		\texttt{mb} & Bottom-quark mass \\
		\texttt{mc} & Charm-quark mass \\
		\texttt{wmass} & W-boson mass \\
		\texttt{zmass} & Z-boson mass \\
		\bottomrule
	\end{longtable}
%\end{table}

%\begin{table}[]
	\begin{longtable}{p{1.5cm}p{12cm}}
		\toprule
		\multicolumn{1}{c}{{\textbf{Section} \texttt{basicjets}}} & \multicolumn{1}{c}{{\textbf{Description}}} \\ 
		\midrule
		\texttt{inclusive} &
		This logical parameter chooses whether the
		calculated cross-section should be inclusive in the number of jets
		found at \NLO{}. An {\em exclusive}
		cross-section contains the same number of jets at next-to-leading
		order as at leading order. An {\em inclusive} cross-section may
		instead contain an extra jet at \NLO{}. \\
		\texttt{algorithm} &
		This specifies the jet-finding algorithm that
		is used, and can take the values
		{\tt ktal} (for the Run II $k_T$-algorithm), {\tt ankt} (for the
		``anti-$k_T$'' algorithm~\cite{Cacciari:2008gp}), {\tt cone} (for
		a midpoint cone algorithm), {\tt hqrk} (for a simplified cone
		algorithm designed for heavy quark processes) and {\tt none} (to
		specify no jet clustering at all). The latter option is only a
		sensible choice when the leading order cross-section is well-defined
		without any jet definition: e.g. the single top process,
		$q{\bar q^\prime} \to t{\bar b}$, which is finite as
		$p_T({\bar b}) \to 0$. \\
		\texttt{ptjetmin}, \texttt{etajetmax} &
		These specify the values
		of $p_{T,{\rm min}}$ and $|\eta|_{\rm max}$ for the
		jets that are found by the algorithm.  \\
		\texttt{etajetmin} &
		Optional parameter for setting a minimum jet rapidity $|\eta|_{\rm min}$. \\
		\texttt{Rcutjet} &
		If the final state of the chosen process contains
		either quarks or gluons then for each event an attempt will be made
		to form them into jets. For this it is necessary to define the
		jet separation $\Delta R=\sqrt{{\Delta \eta}^2 + {\Delta \phi}^2}$
		so that after jet combination, all jet pairs are separated by
		$\Delta R >$~{\tt Rcutjet}.\\
		\bottomrule
	\end{longtable}
%\end{table}


%\begin{table}[]
	\begin{longtable}{p{1.5cm}p{12cm}}
		\toprule
		\multicolumn{1}{c}{{\textbf{Section} \texttt{masscuts}}} & \multicolumn{1}{c}{{\textbf{Description}}} \\ 
		\midrule
		{\tt m34min}, {\tt m34max}, {\tt m56min}, {\tt m56max}, {\tt m3456min}, {\tt m3456max} &
		These parameters represent a basic set of mass cuts that are be applied
		to the calculated cross-section. The only events that contribute to
		the cross-section will have, for example,
		{\tt m34min} $<$ {\tt m34} $<$ {\tt m34max} where {\tt m34} is the
		invariant mass of particles 3 and 4 that are specified by {\tt nproc}.
		{\tt m34min}~$> 0$ is obligatory for processes which can involve a virtual
		photon, such as {\tt nproc=31}. By default, the maximum settings are set to $\sqrt{s}$.\\
		\bottomrule
	\end{longtable}
%\end{table}
\clearpage


%\begin{table}[]
	\begin{longtable}{p{1.5cm}p{12cm}}
		\toprule
		\multicolumn{1}{c}{{\textbf{Section} \texttt{cuts}}} & \multicolumn{1}{c}{{\textbf{Description}}} \\ 
		\midrule
		\texttt{makecuts} &
		If this parameter is set to {\tt .false.} then
		no additional cuts are applied to the events and the remaining
		parameters in this section are ignored. Otherwise, events will
		be rejected according to a set of cuts that is specified below.
		Further options may be implemented by editing {\tt src/User/gencuts\_user.f90}. \\
		
		{\tt ptleptmin, etaleptmax} & These specify the values
		of $p_{T,{\rm min}}$ and $|\eta|_{\rm max}$ for one of the leptons produced
		in the process. One can also introduce optional settings \texttt{ptleptmin}
		and \texttt{etaleptmin}. \\
		
		{\tt etaleptveto} & This should be specified as a pair of double
		precision numbers that indicate a rapidity range that should be excluded
		for the lepton that passes the above cuts. \\
		
		{\tt ptminmiss} & Specifies the minimum missing transverse
		momentum (coming from neutrinos). \\
		
		{\tt ptlept2min}, \texttt{etalept2max} & These specify
		the values of $p_{T,{\rm min}}$ and $|\eta|_{\rm max}$ for the remaining
		leptons in the process. This allows for staggered cuts where, for
		instance, only one lepton is required to be hard and central.
		One can also introduce optional settings \texttt{ptlept2max} and
		\texttt{etalept2min}. \\
		
		{\tt etalept2veto} & This should be specified as a pair of double
		precision numbers that indicate a rapidity range that should be excluded
		for the remaining leptons. \\
		
		{\tt m34transmin} & For general processes, this specifies the
		minimum transverse mass of particles 3 and 4,
		\begin{equation}
		\mbox{general}: \quad 2 p_T(3) p_T(4) \left( 1 - \frac{\vec{p_T}(3) \cdot \vec{p_T}(4)}{p_T(3) p_T(4)} \right) 
		> {\texttt{m34transmin}} 
		\end{equation}
		For the $W(\to \ell \nu)\gamma$ process the role of this cut changes, to become
		instead a cut on the transverse cluster mass of the $(\ell\gamma,\nu)$ system,
		\begin{eqnarray}
		W\gamma: && \left[ \sqrt{m_{\ell\gamma}^2 + |\vec{p_T}(\ell)+\vec{p_T}(\gamma)|^2} + p_T(\nu) \right]^2
		\nonumber \\ &&
		-|\vec{p_T}(\ell)+\vec{p_T}(\gamma)+\vec{p_T}(\nu)|^2 >  {\texttt{m34transmin}}^2
		\end{eqnarray}
		For the $Z\gamma$ process this parameter specifies a simple invariant mass cut,
		\begin{equation}
		Z\gamma: \quad m_{Z\gamma} > {\texttt{m34transmin}}
		\end{equation}
		A final mode of operation applies to the $W\gamma$ process and is triggered by a negative value
		of {\texttt{m34transmin}}. This allows simple access to the cut that was employed in v6.0 of the code:
		\begin{eqnarray}
		W\gamma, \mbox{obsolete}: &&
		\left[ p_T(\ell) +  p_T(\gamma) +  p_T(\nu) \right]^2 \nonumber \\ 
		&-&|\vec{p_T}(\ell)+\vec{p_T}(\gamma)+\vec{p_T}(\nu)|^2 > |{\texttt{m34transmin}}|
		\end{eqnarray}
		In each case the screen output indicates the cut that is applied. \\
		
		\bottomrule
	\end{longtable}
%\end{table}
\clearpage
%\begin{table}[]
	\begin{longtable}{p{1.5cm}p{12cm}}
		\toprule
		\multicolumn{1}{c}{{\textbf{Section} \texttt{cuts}}} & \multicolumn{1}{c}{{\textbf{Description}}} \\ 
		\midrule
		{\tt Rjlmin} & Using the definition of $\Delta R$ above,
		requires that all jet-lepton pairs are separated by
		$\Delta R >$~{\tt R(jet,lept)\_min}. \\
		
		{\tt Rllmin} & When non-zero, all lepton-lepton pairs
		must be separated by $\Delta R >$~{\tt R(lept,lept)\_min}. \\
		
		{\tt delyjjmin} & This enforces a pseudo-rapidity
		gap between the two hardest jets $j_1$ and $j_2$, so that:
		$|\eta^{j_1} - \eta^{j_2}| >$~{\tt Delta\_eta(jet,jet)\_min}. \\
		
		{\tt jetsopphem} & If this parameter is set to {\tt .true.},
		then the two hardest jets are required to lie in opposite hemispheres,
		$\eta^{j_1} \cdot \eta^{j_2} < 0$. \\
		
		{\tt lbjscheme} & This integer parameter provides no
		additional cuts when it takes the value {\tt 0}. When equal to
		{\tt 1} or {\tt 2}, leptons are required to lie between the two
		hardest jets. With the ordering $\eta^{j_-} < \eta^{j_+}$ for the
		pseudo-rapidities of jets $j_1$ and $j_2$:
		{\tt lbjscheme = 1} : 
		$\eta^{j_-} < \eta^{\rm leptons} < \eta^{j_+}$;
		{\tt lbjscheme = 2} :
		$\eta^{j_-}+{\tt Rcutjet} < \eta^{\rm leptons} < \eta^{j_+}-{\tt Rcutjet}$. \\
		
		{\tt ptbjetmin, etabjetmax} & If a process involving $b$-quarks is being calculated, then these can
		be used to specify {\em stricter} values of $p_T^{\rm min}$
		and $|\eta|^{\rm max}$ for $b$-jets. Similarly, values for \texttt{ptbjetmax} and \texttt{etabjetmin} can be 
		specified. \\
		\bottomrule
	\end{longtable}
%\end{table}


%\begin{table}[]
	\begin{longtable}{p{3.5cm}p{12cm}}
		\toprule
		\multicolumn{1}{c}{{\textbf{Section} \texttt{photon}}} & \multicolumn{1}{c}{{\textbf{Description}}} \\ 
		\midrule
		{\tt fragmentation} &  This parameter is a logical variable that determines whether the production of photons 
		by a parton 
		fragmentation process is included. If {\tt fragmentation} is set to {\tt .true.} the code uses a a standard 
		cone isolation
		procedure (that includes LO fragmentation contributions in the NLO calculation).
		If {\tt fragmentation} is set to {\tt .false.} the code implements
		a Frixione-style photon cut~\cite{Frixione:1998jh},
		\begin{eqnarray}
		\sum_{i \in R_0} E_{T,i}^j  < \epsilon_h E_{T}^{\gamma} \bigg(\frac{1-\cos{R_{i\gamma}}}{1-\cos{R_0}}\bigg)^{n} 
		\;.
		\label{frixeq}
		\end{eqnarray}
		In this equation, $R_0$, $\epsilon_h$ and $n$ are defined by {\tt cone\_ang}, {\tt epsilon\_h} 
		and {\tt n\_pow}  respectively (see below).
		$E_{T,i}^{j}$ is the transverse energy of a parton, $E_{T}^\gamma$ is the
		transverse energy of the photon and $R_{i\gamma}$ is the separation between the photon and the parton using the 
		usual definition 
		$R=\sqrt{\Delta\phi^2+\Delta\eta^2}$. $n$ is an integer parameter which by default is set to~1. \\
		
		{\tt fragmentation\_set} & A length eight character variable that is used to choose the particular photon 
		fragmentation set.
		Currently implemented fragmentation functions can be called with `{\tt BFGSet\_I}', `{\tt 
			BFGSetII}'~\cite{Bourhis:1997yu} or `{\tt GdRG\_\_LO}'~\cite{GehrmannDeRidder:1998ba}. \\
		
		{\tt fragmentation\_scale} & A double precision variable that will be used to choose the scale 
		at which the photon fragmentation is evaluated. \\
		
		{\tt gammptmin} & This specifies the value
		of $p_T^{\rm min}$ for the photon with the largest transverse momentum.
		Note that this cut, together with all the photon cuts specified in this section
		of the input file, are applied even if {\tt makecuts} is set to {\tt .false.}.
		One can also add an entry for \texttt{gammptmax} to cut on a range. \\
		
		{\tt gammrapmax} & This specifies the value
		of $|y|^{\rm max}$ for any photons produced in the process. One can also add an entry
		for \texttt{gammrapmin} to cut on a range. \\
		
		{\tt gammpt2} and {\tt gammpt3} & These specify the values
		of $p_T^{\rm min}$ for the second and third photons, ordered by $p_T$. \\
		
		{\tt Rgalmin} & Using the usual definition of $\Delta R$,
		this requires that all photon-lepton pairs are separated by
		$\Delta R >$~{\tt Rgalmin}. This parameter must be non-zero
		for processes in which photon radiation from leptons is included. \\
		
		{\tt Rgagamin} & Using the usual definition of $\Delta R$,
		this requires that all photon pairs are separated by
		$\Delta R >$~{\tt Rgagamin}. \\
		
		{\tt Rgajetmin} & Using the usual definition of $\Delta R$,
		this requires that all photon-jet pairs are separated by
		$\Delta R >$~{\tt Rgajetmin}. \\
		
		{\tt cone\_ang} & Fixes the cone size ($R_0$) for photon isolation.
		This cone is used in both forms of isolation. \\
		
		{\tt epsilon\_h} & This cut controls the amount of radiation allowed in cone when  {\tt fragmentation} is set 
		to 
		{\tt .true.}. If  {\tt epsilon\_h} $ < 1$ then the photon is isolated using
		$\sum_{\in R_0} E_T{\rm{(had)}} < \epsilon_h \, p^{\gamma}_{T}.$ Otherwise {\tt epsilon\_h}  $ > 1$ sets 
		$E_T(max)$ in  $\sum_{\in R_0} E_T{\rm{(had)}} < E_T(max)$.  
		If the user wishes to always use a scaling or fixed isolation cut, independent of the value of {\tt 
			epsilon\_h}, the routine
		{\tt src/Cuts/iso.f} may be edited and the value of the variable {\tt imode} changed according to the comments.
		When {\tt fragmentation} is set to {\tt .false.}, $\epsilon_h$ controls the amount of hadronic energy allowed 
		inside the cone using the
		Frixione isolation prescription (see above, Eq.~(\ref{frixeq})). \\
		
		{\tt n\_pow} & When using the Frixione isolation prescription, the exponent $n$ in Eq.~(\ref{frixeq}). \\
		
		\bottomrule
	\end{longtable}
%\end{table}

%\begin{table}[]
	\begin{longtable}{p{1.5cm}p{12cm}}
		\toprule
		\multicolumn{1}{c}{{\textbf{Section} \texttt{histogram}}} & \multicolumn{1}{c}{{\textbf{Description}}} \\ 
		\midrule
		\texttt{writetop} & Write output histograms suitable as input for top-drawer. \\
		\texttt{writetxt} & Write output histograms as whitespace-separated columns. \\
		\bottomrule
	\end{longtable}
%\end{table}

%\begin{table}[]
	\begin{longtable}{p{1.5cm}p{12cm}}
		\toprule
		\multicolumn{1}{c}{{\textbf{Section} \texttt{integration}}} & \multicolumn{1}{c}{{\textbf{Description}}} \\ 
		\midrule
		\texttt{usesobol} & When \texttt{.true.} and the number of \MPI{} processes is a power of two, the Sobol 
		sequence is used, see ref.~\cite{MCFM9}, otherwise the {\abbrev MT19937} pseudo random number generator. \\
		\texttt{seed} & Initialization seed for {\abbrev MT19937} pseudo random number generator. \\
		\texttt{precisiongoal} & Relative precision goal for the integration. \\
		\texttt{readin} & When \texttt{.true.} the automatically written snapshot from a previous run will be read-in
		to resume the integration. \\
		\texttt{writeintermediate} & When \texttt{.true.} histograms are written after each Vegas iteration. \\
		\texttt{warmupprecisiongoal} & Sets the relative precision goal for the warmup run. Unless this precision
		is reached, the number of calls for the warmup is increased. \\
		\texttt{warmupchisqgoal} & Sets the $\chi^2$ per iteration goal for the warmup run. Unless the 
		$\chi^2/\text{it.}$ of the warmup is below this target, the number of calls for the warmup is increased. \\
		\bottomrule
	\end{longtable}
%\end{table}

\clearpage

\subsection{Process specific options}

%\begin{table}[]
	\begin{longtable}{p{1.5cm}p{12cm}}
		\toprule
		\multicolumn{1}{c}{{\textbf{Section} \texttt{singletop}}} & \multicolumn{1}{c}{{\textbf{Description}}} \\ 
		\midrule
		\texttt{c\_phiq} & Sets real Wilson coefficient of $\Qone$ for processes 164 and 169. See \cref{subsec:offstop} 
		and ref.~\cite{Neumann:2019kvk}. \\
		\texttt{c\_phiphi} & Sets real and imaginary part of the $\Qtwo$ Wilson coefficient. \\
		\texttt{c\_tw} & Sets real and imaginary part of the $\Qthree$ Wilson coefficient. \\
		\texttt{c\_bw} & Sets real and imaginary part of the $\Qfour$ Wilson coefficient. \\
		\texttt{c\_tg} & Sets real and imaginary part of the $\Qsix$ Wilson coefficient. \\
		\texttt{c\_bg} & Sets real and imaginary part of the $\Qseven$ Wilson coefficient. \\
		\texttt{lambda} & Scale $\Lambda$, see \cref{subsec:offstop} and ref.~\cite{Neumann:2019kvk}. \\
		\texttt{enable\_lambda4} & Enable contributions of order $1/\Lambda^4$ when set to \texttt{.true.}. \\
		\texttt{disable\_sm} & When set to \texttt{.true.} the pure \SM{} contributions are disabled, and just
		the \SM{}-\EFT{} interference and \EFT{} contributions are calculated. \\
		\texttt{mode\_anomcoup} & When set to \texttt{.true.} at \LO{} one can reproduce results obtained
			without power counting as in the anomalous couplings approach, see \cref{subsec:offstop} and 
			ref.~\cite{Neumann:2019kvk}. \\
		\bottomrule
	\end{longtable}
%\end{table}

%\begin{table}[]
	\begin{longtable}{p{1.5cm}p{12cm}}
		\toprule
		\multicolumn{1}{c}{{\textbf{Section} \texttt{anom\_wz}}} & \multicolumn{1}{c}{{\textbf{Description}}} \\ 
		\midrule
		{\tt enable} &  Boolean flag to enable anomalous W-boson and Z-boson coupling contributions for certain 
		processes. 	False has the same effect as setting all anomalous couplings to zero, but additionally skips 
		computation of anomalous coupling code parts. \\
		 & \\
		{\tt delg1\_z} & $\Delta g_1^Z$ {\it See section~\ref{subsec:diboson}.} \\
		{\tt delk\_z} & $\Delta\kappa^Z$ {\it See section~\ref{subsec:diboson}.} \\
		{\tt delk\_g} & $\Delta\kappa^\gamma$ {\it See sections~\ref{subsec:diboson} and~\ref{subsec:wgamma}.} \\
		{\tt lambda\_z} & $\Lambda^Z$ {\it See section~\ref{subsec:diboson}.} \\
		{\tt lambda\_g} & $\Lambda^\gamma$ {\it See sections~\ref{subsec:diboson} and~\ref{subsec:wgamma}.} \\
		 & \\
		{\tt h1Z} & $h_1^Z$ {\it Anomalous couplings for $Z\gamma$ process at \NNLO{}. See 
		section~\ref{subsec:zgamma}.} \\
		{\tt h1gam} & $h_1^\gamma$ {\it See section~\ref{subsec:zgamma}.} \\
		{\tt h2Z} & $h_2^Z$ {\it See section~\ref{subsec:zgamma}.} \\
		{\tt h2gam} & $h_2^\gamma$ {\it See section~\ref{subsec:zgamma}.} \\
		{\tt h3Z} & $h_3^Z$ {\it See section~\ref{subsec:zgamma}.} \\
		{\tt h3gam} & $h_3^\gamma$ {\it See section~\ref{subsec:zgamma}.} \\
		{\tt h4Z} & $h_4^Z$ {\it See section~\ref{subsec:zgamma}.} \\
		{\tt h4gam} & $h_4^\gamma$ {\it See section~\ref{subsec:zgamma}.} \\
		 & \\
		{\tt tevscale} & Form-factor scale, in TeV {\it See section~\ref{subsec:diboson}.} 
		No form-factors are applied to the anomalous couplings if this value is negative. \\
		\bottomrule
	\end{longtable}
%\end{table}

\clearpage
%\begin{table}[]
	\begin{longtable}{p{1.5cm}p{12cm}}
		\toprule
		\multicolumn{1}{c}{{\textbf{Section} \texttt{wz2jet}}} & \multicolumn{1}{c}{{\textbf{Description}}} \\ 
		\midrule
		\texttt{Qflag} &
		This only has an effect when running a
		$W+2$~jets or $Z+2$~jets process. Please see section~\ref{subsec:w2jets}
		below. \\
		\texttt{Gflag} &
		This only has an effect when running a
		$W+2$~jets or $Z+2$~jets process. Please see section~\ref{subsec:w2jets}
		below. \\
		\bottomrule
	\end{longtable}
%\end{table}

%\begin{table}[]
	\begin{longtable}{p{1.5cm}p{12cm}}
		\toprule
		\multicolumn{1}{c}{{\textbf{Section} \texttt{hjetmass}}} & \multicolumn{1}{c}{{\textbf{Description}}} \\ 
		\midrule
		{\tt mtex} &
		Sets the order $k=0,2,4$ of the $1/m_t^k$ expansion for virtual corrections in the $H+$jet process 200. See 
		section \ref{subsec:hjetma}. \\
		\bottomrule
	\end{longtable}
%\end{table}

%\begin{table}[]
	\begin{longtable}{p{1.5cm}p{12cm}}
		\toprule
		\multicolumn{1}{c}{{\textbf{Section} \texttt{anom\_higgs}}} & \multicolumn{1}{c}{{\textbf{Description}}} \\ 
		\midrule
		{\tt hwidth\_ratio} & For processes {\tt 123}--{\tt 126}, {\tt 128}--{\tt 133} only,
		this variable provides a rescaling of the width of the Higgs boson.  Couplings are rescaled such that the
		corresponding cross section close to the Higgs boson peak is unchanged.  Further details of this procedure are 
		given in
		{\tt arXiv:1311.3589}. \\
		\texttt{cttH, cWWH} & See {\tt arXiv:1311.3589}. \\
		\bottomrule
	\end{longtable}
%\end{table}

%\begin{table}[]
	\begin{longtable}{p{1.5cm}p{12cm}}
		\toprule
		\multicolumn{1}{c}{{\textbf{Section} \texttt{extra}}} & \multicolumn{1}{c}{{\textbf{Description}}} \\ 
		\midrule
		{\tt debug} &
		A logical variable which can be used during a 
		debugging phase to mandate special behaviours. 
		Passed by common block {\tt common/debug/debug}. \\
		
		{\tt verbose} &
		A logical variable which can be used during a debugging phase to write 
		special information. Passed in common block {\tt common/verbose/verbose}. \\
		
		{\tt new\_pspace} &
		A logical variable which can be used during a debugging phase to test alternative versions of the phase space.
		Passed in common block {\tt common/new\_pspace/new\_pspace}. \\
		
		{\tt spira} & 
		A  variable. If {\tt spira} is true, we calculate the 
		width of the Higgs boson by interpolating from a table
		calculated using the NLO code of M. Spira.	Otherwise the LO value valid for low Higgs masses only is used. \\
		
		{\tt noglue} &
		A logical variable. 
		The default value is false. If set to true, no processes
		involving initial gluons are included. \\
		{\tt ggonly} &
		A logical variable. 
		The default value is false. If set to true, 
		only the processes
		involving initial gluons in both hadrons are included.\\
		{\tt gqonly} &
		The default value is false. If set to true, 
		only the processes
		involving an initial gluon in one hadron and an initial quark
		or antiquark in the other hadron (or vice versa) are included.\\
		{\tt omitgg} &
		A logical variable. 
		The default value is false. If set to true, the gluon-gluon
		initial state is not included.\\
		
		{\tt clustering} &
		This logical parameter determines whether clustering is performed to yield
		jets. Only during a debugging phase should this variable be set to false. \\
		
		{\tt colourchoice} &
		If colourchoice=0, all colour structure are included ($W,Z+2$~jets).
		If colourchoice=1, only the leading 
		colour structure is included ($W,Z+2$~jets). \\
		
		{\tt rtsmin} &
		A minimum value of $\sqrt{s_{12}}$, which ensures that the invariant mass
		of the incoming partons can never be less than {\tt rtsmin}. \\
		
		
		{\tt cutoff} & Cutoff according to \texttt{src/Need/smallnew.f} and
		\texttt{src/Need/smalltau.f} \\
		\bottomrule
	\end{longtable}
%\end{table}

%\begin{table}[]
	\begin{longtable}{p{1.5cm}p{12cm}}
		\toprule
		\multicolumn{1}{c}{{\textbf{Section} \texttt{dipoles}}} & \multicolumn{1}{c}{{\textbf{Description}}} \\ 
		\midrule
		{\tt aii} &
		A double precision variable which can be used to
		limit the kinematic range for the subtraction of initial-initial dipoles
		as suggested by Trocsanyi and Nagy~\cite{Nagy:2003tz}.   
		The value {\tt aii=1} corresponds 
		to standard Catani-Seymour subtraction.\\
		{\tt aif} &
		A double precision variable which can be used to
		limit the kinematic range for the subtraction of initial-final dipoles
		as suggested by Trocsanyi and Nagy~\cite{Nagy:2003tz}.   
		The value {\tt afi=1} corresponds 
		to standard Catani-Seymour subtraction.\\
		{\tt afi} &
		A double precision variable which can be used to
		limit the kinematic range for the subtraction of final-initial dipoles
		as suggested by Trocsanyi and Nagy~\cite{Nagy:2003tz}.   
		The value {\tt afi=1} corresponds 
		to standard Catani-Seymour subtraction.\\
		{\tt aff} &
		A double precision variable which can be used to
		limit the kinematic range for the subtraction of final-final dipoles
		as suggested by Trocsanyi and Nagy~\cite{Nagy:2003tz}.   
		The value {\tt aff=1} corresponds 
		to standard Catani-Seymour subtraction.\\
		{\tt bfi} &
		A double precision variable which can be used to
		limit the kinematic range for the subtraction of final-initial dipoles
		in the photon fragmentation case.\\
		{\tt bff} &
		A double precision variable which can be used to
		limit the kinematic range for the subtraction of final-final dipoles
		in the photon fragmentation case.\\
		\bottomrule
	\end{longtable}
%\end{table}

\subsection{User modifications to the code}


Modifying the plotting routines in the files \texttt{src/User/nplotter*.f} allows for modification of the pre-defined 
histograms and addition of any number of arbitrary observables. The routine \texttt{gencuts\_user} can be adjusted  in 
the file
\texttt{src/User/gencuts\_user.f90} for additional cuts after the jet algorithm has performed the 
clustering. In the same file the routine \texttt{reweight\_user} can be modified to include a manual re-weighting
for all integral contributions. This can be used to obtain improved uncertainties in, for example, tails of 
distributions.
One example is included in the subdirectory \texttt{examples}, where the \texttt{reweight\_user} function approximately
flattens the Higgs transverse momentum distribution, leading to equal relative uncertainties even in the tail at 
\SI{1}{TeV}.

\label{user}
\begin{itemize}
	\item {\tt subroutine nplotter\_user(pjet, wt,wt2, nd)}
	This subroutine is called to allow the user to bin their own       
	histograms. Variables passed to this routine:
	
	\begin{itemize}
		\item        p:  4-momenta of incoming partons(i=1,2), outgoing leptons 
		and                                                             
		jets in the format p(i,4) with the particles 
		numbered                                                                  
		according to the input file and components labelled 
		by                                                                 
		(px,py,pz,E).  
		
		
		\item        wt:  weight of this 
		event                                                                                                   
		
		
		\item       wt2:  weight$^2$ of this 
		event                                                                                                 
		
		
		\item        nd:  an integer specifying the dipole number of this 
		contribution                                                           
		(if applicable), otherwise equal to zero.
	\end{itemize}
	
\end{itemize}
